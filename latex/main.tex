\documentclass[12pt]{article}

\usepackage[polish]{babel}
\usepackage[T1]{fontenc}
\usepackage[utf8]{inputenc}
\usepackage{listings}
\usepackage{xcolor}
\usepackage{csquotes}

\lstset{
  basicstyle=\ttfamily\small,
  frame=single,
  numbers=left,
  numberstyle=\tiny,
  breaklines=true
}

\usepackage{graphicx}
\graphicspath{{img/}}

\usepackage{geometry}
\geometry{margin=2.5cm}

\usepackage{setspace}
\onehalfspacing{}

% Bibliografia
\usepackage[backend=biber,style=ieee]{biblatex}
\addbibresource{bib/references.bib}

\begin{document}

\tableofcontents
\newpage

\section{Wstęp}
\chapter{Wstęp}
\label{ch:wstep}

wstep

%%%%%%%%%%%%%%%%%%%%%%%%%%%%%%%%%%%%%%%%%%%%%%%%
\section{Podrozdzial1}

bla

\section{Podrozdzial2}

bla

\section{Cyfrowy generator liczb losowych}

Celem niniejszego rozdziału jest omówienie koncepcji cyfrowego generatora
liczb losowych, ze szczególnym uwzględnieniem źródeł entropii oraz
sprzętowej realizacji układu. Przedstawiono rolę oscylatora
pierścieniowego jako elementu wprowadzającego niedeterministyczność oraz
znaczenie platformy FPGA dla jakości i powtarzalności implementacji.

\subsection{Oscylator pierścieniowy jako źródło entropii}
Oscylator pierścieniowy (ang. \emph{ring oscillator}) jest jednym z
najprostszych układów generujących sygnał okresowy w technologiach
cyfrowych. Składa się z nieparzystej liczby inwerterów połączonych
szeregowo w zamkniętą pętlę, gdzie wyjście ostatniego elementu jest
podłączone do wejścia pierwszego.

Zasada działania jest intuicyjna: sygnał logiczny krążący w pętli jest
kolejno odwracany przez każdy inwerter. Ponieważ liczba inwerterów jest
nieparzysta, układ nie może osiągnąć stabilnego stanu logicznego.
Zamiast tego powstaje ciągła oscylacja pomiędzy stanem wysokim i niskim.
Częstotliwość tych oscylacji zależy głównie od sumy opóźnień propagacji
poszczególnych bramek.

Istotną cechą oscylatora pierścieniowego jest jego podatność na zjawiska
fizyczne zachodzące w strukturze krzemowej. Szumy termiczne, fluktuacje
napięcia zasilania, zmiany temperatury oraz lokalne różnice
technologiczne powodują losowe zmiany opóźnień propagacji. Skutkiem tego
jest nieregularność okresu sygnału, określana jako jitter, który stanowi
praktyczne źródło entropii w wielu konstrukcjach TRNG.\cite{sunar2006provably,saarinen2021entropy}

Losowy charakter jittera sprawia, że oscylator pierścieniowy może pełnić
rolę źródła entropii w cyfrowych generatorach liczb losowych. W praktyce
entropia pozyskiwana jest poprzez próbkowanie sygnału oscylatora lub
porównywanie faz i częstotliwości wielu niezależnych oscylatorów.
Dobór parametrów próbkowania i architektury ma bezpośredni wpływ na
entropię na bit i odporność na zakłócenia.\cite{ma2014entropy}

Dzięki prostocie, niewielkim wymaganiom sprzętowym i dobrej skalowalności
w nowoczesnych procesach technologicznych, oscylatory pierścieniowe są
powszechnie wykorzystywane jako elementy generatorów liczb losowych w
układach FPGA, mikrokontrolerach oraz systemach kryptograficznych.


\subsection{Implementacja generatora w strukturach FPGA}
Układy FPGA (ang. \emph{Field Programmable Gate Array}) stanowią
elastyczną platformę do implementacji cyfrowych generatorów liczb
losowych, w tym konstrukcji bazujących na jitterze oscylatorów
pierścieniowych. Architektura oparta na programowalnych blokach logicznych
i rekonfigurowalnych połączeniach umożliwia bezpośrednią realizację
struktur sprzętowych i ich równoległe zwielokrotnienie.\cite{peetermans2019portable}

W przeciwieństwie do rozwiązań programowych, implementacja generatora
w strukturze FPGA pozwala na kontrolę nad topologią, sposobem próbkowania
oraz logiką post-processingu. Jest to istotne, ponieważ zarówno jakość
źródła entropii, jak i mechanizm jej ekstrakcji wpływają na końcową
losowość strumienia bitów.\cite{ma2014entropy}

W praktyce platformy FPGA są często wybierane w zastosowaniach
wymagających wysokiej przepustowości i deterministycznej latencji,
a także tam, gdzie istotna jest przenośność rozwiązania pomiędzy
różnymi rodzinami układów. Przykładem są współczesne konstrukcje TRNG
projektowane tak, aby zachowywać właściwości statystyczne niezależnie
od producenta i szczegółów technologii.\cite{matuszewski2024startstop}


\section{Implementacja generatora na platformie FPGA}

W niniejszym rozdziale przedstawiono praktyczną implementację cyfrowego
generatora liczb losowych wykorzystującego oscylator pierścieniowy jako
źródło entropii. Opis obejmuje wybór środowiska projektowego oraz
platformy sprzętowej, a następnie proces syntezy, implementacji i
uruchomienia układu na fizycznej płytce FPGA. Celem rozdziału jest
przedstawienie pełnego przepływu projektowego, od opisu sprzętowego do
działającej realizacji.

\subsection{Środowisko projektowe Vivado i wybór platformy sprzętowej}
Do realizacji projektu wybrano środowisko Xilinx Vivado Design Suite,
które stanowi podstawowe narzędzie do projektowania, syntezy oraz
implementacji układów FPGA firmy Xilinx. Vivado umożliwia opis logiki
sprzętowej w językach HDL, przeprowadzenie syntezy logicznej,
rozmieszczenia i trasowania zasobów, a także generację pliku konfiguracyjnego
(bitstream) i programowanie układu docelowego.

Istotną zaletą środowiska Vivado jest pełna kontrola nad strukturą
projektowanego układu oraz dostęp do raportów czasowych i zasobowych,
co ma kluczowe znaczenie w przypadku oscylatorów pierścieniowych.
Niewielkie różnice w opóźnieniach propagacji bramek oraz sposób
rozmieszczenia logiki w strukturze FPGA mają bezpośredni wpływ na
właściwości generowanego sygnału i poziom entropii.

Jako platformę sprzętową wybrano płytkę rozwojową Arty S7--25 z układem
Xilinx Spartan-7. Jest to nowoczesny układ FPGA przeznaczony do aplikacji
o umiarkowanej złożoności, oferujący dobrą relację dostępnych zasobów
logicznych do poboru mocy i kosztu. Rodzina Spartan-7 zapewnia stabilne
parametry czasowe oraz powtarzalne zachowanie struktur logicznych, co
jest istotne w kontekście implementacji generatorów liczb losowych.

Dodatkowym argumentem przemawiającym za wyborem płytki Arty S7--25 jest jej
szerokie wsparcie narzędziowe oraz dostępność dokumentacji i przykładów
projektowych. Płytka posiada standardowy interfejs JTAG do programowania
i debugowania, co umożliwia wygodne testowanie implementacji oraz
obserwację zachowania układu w warunkach rzeczywistych.


\subsection{Implementacja oscylatora pierścieniowego w języku VHDL}
W niniejszym podrozdziale przedstawiono implementację testową oscylatora 
pierścieniowego oraz nadrzędnego modułu generatora liczb losowych,
opracowaną w języku VHDL i przeznaczoną do syntezy w strukturze FPGA.

\begin{lstlisting}[language=VHDL,
caption={Moduł oscylatora pierścieniowego (ring\_oscillator.vhdl)},
label={lst:ring_oscillator}]
library IEEE;
use IEEE.STD_LOGIC_1164.ALL;

entity ring_oscillator is
    generic ( INVERTERS_NUM : integer := 5 );                        
    port    ( ro_out : out std_logic       );                              
end ring_oscillator;

architecture rtl of ring_oscillator is
    signal    chain               : std_logic_vector(INVERTERS_NUM-1 downto 0);
    attribute dont_touch          : boolean;                              
    attribute dont_touch of chain : signal is true;                       
begin
    chain(0) <= not chain(INVERTERS_NUM-1);                              
    gen_chain : for i in 1 to INVERTERS_NUM-1 generate 
        chain(i) <= not chain(i-1);
    end generate;
    ro_out <= chain(INVERTERS_NUM-1);
end rtl;
\end{lstlisting}

Listing~\ref{lst:ring_oscillator} przedstawia prostą, parametryzowaną implementację
oscylatora pierścieniowego w języku VHDL. Zastosowane rozwiązanie odwzorowuje klasyczną 
ideę oscylatora pierścieniowego znaną z elektroniki cyfrowej: nieparzystą liczbę inwerterów
połączonych szeregowo w pętlę zwrotną, w której brak stabilnego stanu logicznego wymusza ciągłe przełączanie.
Rdzeniem projektu jest wektor sygnałów \texttt{chain} którego długość jest określona parametrem
generycznym \texttt{INVERTERS\_NUM}. Każdy element tego wektora odpowiada jednemu stopniowi inwersji.
Pierwszy stopień otrzymuje na wejście zanegowany sygnał z ostatniego elementu łańcucha,
co zamyka pętlę sprzężenia zwrotnego. Kolejne stopnie realizują prostą inwersję sygnału
z poprzedniego etapu.
Kluczowym elementem implementacyjnym jest użycie atrybutu \texttt{dont\_touch}, przypisanego do 
sygnału \texttt{chain}. Atrybut ten informuje narzędzia syntezy\cite{amd_vivado_donttouch}, 
że pętla kombinacyjna nie jest błędem projektowym, lecz zamierzoną strukturą funkcjonalną. 
Bez tej deklaracji optymalizator mógłby usunąć lub uprościć logikę, traktując ją jako zbędną 
lub niepoprawną. W praktyce atrybut zwiększa szansę zachowania fizycznej struktury pierścienia 
w docelowym układzie FPGA, co jest warunkiem powstania oscylacji.
Sygnał wyjściowy \texttt{ro\_out} jest bezpośrednio pobierany z ostatniego stopnia łańcucha inwerterów.
Jego częstotliwość zależy od liczby stopni inwersji oraz opóźnień propagacji w danej technologii 
sprzętowej. Dzięki użyciu parametru generycznego możliwa jest łatwa modyfikacja liczby inwerterów,
a tym samym regulacja częstotliwości oscylatora bez ingerencji w strukturę kodu.
Całość architektury ma charakter czysto kombinacyjny i świadomie narusza klasyczne zasady
projektowania synchronicznego, co czyni ją przykładem konstrukcji niskopoziomowej,
silnie zależnej od właściwości fizycznych układu docelowego, a nie tylko od opisu logicznego.

\begin{lstlisting}[language=VHDL,
caption={Moduł generatora liczb losowych oparty na oscylatorach pierścieniowych (\texttt{trng\_ro.vhdl})},
label={lst:trng_ro}]
library IEEE;
use IEEE.STD_LOGIC_1164.ALL;

entity trng_ro is
    generic (
        NUM_RO    : integer := 8;
        RO_STAGES : integer := 5
    );
    port (
        clk        : in  std_logic;
        rst_n      : in  std_logic;
        random_bit : out std_logic
    );
end trng_ro;

architecture rtl of trng_ro is

    component ring_oscillator is
        generic ( INVERTERS_NUM : integer := 5 );
        port    ( ro_out        : out std_logic );
    end component;

    signal ro_vector : std_logic_vector(NUM_RO-1 downto 0);
    signal xor_raw   : std_logic;
    signal sync_1    : std_logic;
    signal sync_2    : std_logic;

    attribute dont_touch              : boolean;
    attribute dont_touch of ro_vector : signal is true;

    function xor_reduce(s : std_logic_vector) return std_logic is
        variable r : std_logic := '0';
    begin
        for i in s'range loop
            r := r xor s(i);
        end loop;
        return r;
    end function;

begin
    gen_ro : for i in 0 to NUM_RO-1 generate
        ro_inst : ring_oscillator
            generic map ( INVERTERS_NUM => RO_STAGES )
            port map    ( ro_out        => ro_vector(i) );
    end generate;

    xor_raw <= xor_reduce(ro_vector);

    sample_and_output_random_bit : process(clk, rst_n)
    begin
        if rst_n = '0' then
            sync_1     <= '0';
            sync_2     <= '0';
            random_bit <= '0';
        elsif rising_edge(clk) then
            sync_1     <= xor_raw;
            sync_2     <= sync_1;
            random_bit <= sync_2;
        end if;
    end process;

end rtl;
\end{lstlisting}

Listing~\ref{lst:trng_ro} przedstawia nadrzędny moduł generatora liczb losowych opartego na zespole
oscylatorów pierścieniowych. Moduł realizuje architekturę TRNG, w której źródłem entropii są 
fluktuacje czasowe i jitter wielu niezależnych oscylatorów, a ich próbki są synchronizowane 
zegarem systemowym.

Parametry generyczne \texttt{NUM\_RO} oraz \texttt{RO\_STAGES} umożliwiają odpowiednio konfigurację 
liczby instancji oscylatorów pierścieniowych oraz liczby stopni inwersji w każdym z nich. Pozwala 
to na łatwe dostosowanie charakterystyk generatora, takich jak poziom entropii czy pobór zasobów 
logicznych, do wymagań aplikacji oraz technologii docelowej.

Wyjścia wszystkich oscylatorów są gromadzone w wektorze \texttt{ro\_vector}, który został oznaczony 
atrybutem \texttt{dont\_touch}. Zapobiega to ingerencji narzędzi syntezy w strukturę fizyczną 
zespołu oscylatorów, co ma kluczowe znaczenie dla zachowania ich niezależności czasowej oraz 
właściwości losowych.

W celu wstępnego „wybielenia” sygnału losowego zastosowano funkcję redukującą XOR, która łączy 
stany logiczne wszystkich oscylatorów w pojedynczy sygnał \texttt{xor\_raw}. Takie podejście 
zmniejsza wpływ ewentualnych korelacji pomiędzy pojedynczymi oscylatorami i poprawia statystyczne 
właściwości wyjścia.

Próbowanie sygnału losowego odbywa się w procesie synchronicznym względem zegara \texttt{clk}. 
Dwustopniowy rejestr synchronizujący pełni podwójną rolę: pierwszy przerzutnik próbuje 
asynchroniczny sygnał pochodzący z oscylatorów, natomiast drugi stabilizuje jego wartość, 
ograniczając propagację metastabilności do dalszej logiki. Wyjście generatora \texttt{random\_bit} 
jest pobierane z drugiego stopnia synchronizacji.

Przedstawiona architektura stanowi typowy przykład konstrukcji TRNG silnie zależnej od właściwości 
fizycznych układu FPGA, w której losowość nie wynika z algorytmu, lecz z niedeterministycznych 
zjawisk czasowych zachodzących w strukturze sprzętowej.

\subsection{Implementacja protokołu UART do transmisji ciągu losowych bitow}
W niniejszym podrozdziale przedstawiono implementację protokołu UART, przeznaczoną do 
transmisji ciągu losowych bitów generowanych przez moduł \texttt{trng\_ro}. Celem tego etapu 
projektu jest umożliwienie niezależnej, asynchronicznej komunikacji pomiędzy układem FPGA a 
systemem zewnętrznym, w którym dane losowe mogą być dalej przetwarzane lub analizowane.

\begin{lstlisting}[language=VHDL,
caption={Moduł bufora UART (uart\_buff.vhdl)},
label={lst:uart_buff}]
library IEEE;
use IEEE.STD_LOGIC_1164.ALL;

entity uart_buff is
    generic (
        
    );
    port (
        
    );
end uart_buff;
\end{lstlisting}

Listing~\ref{lst:uart_buff} przedstawia architekturę pośrednią. Z uwagi na bitowy charakter 
wyjścia generatora \texttt{trng\_ro} moduł \texttt{uart\_buff} stopniowo agreguje pojedyncze bity
do postaci bajtów, zgodnej z formatem transmisji UART. Rozwiązanie to oddziela 
warstwę generacji entropii od warstwy komunikacyjnej, upraszczając zarówno projekt, jak i 
późniejszą weryfikację poszczególnych bloków funkcjonalnych.
Implementacja została podzielona na dwa współpracujące moduły: bufor danych odpowiedzialny 
za gromadzenie bitów losowych oraz nadajnik UART realizujący właściwy proces transmisji 
szeregowej. Takie rozdzielenie funkcjonalności pozwala na zachowanie czytelnej struktury 
projektu oraz ułatwia ewentualną rozbudowę lub modyfikację parametrów transmisji w dalszych 
etapach pracy.
W kolejnych fragmentach podrozdziału zaprezentowano implementację poszczególnych modułów 
w języku VHDL, a następnie omówiono ich rolę w kontekście całego toru transmisji danych losowych.


\newpage
\printbibliography{}

\end{document}
