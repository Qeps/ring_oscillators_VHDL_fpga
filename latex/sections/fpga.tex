Układy FPGA (ang. \emph{Field Programmable Gate Array}) stanowią
elastyczną platformę do implementacji cyfrowych generatorów liczb
losowych, w tym konstrukcji bazujących na jitterze oscylatorów
pierścieniowych. Architektura oparta na programowalnych blokach logicznych
i rekonfigurowalnych połączeniach umożliwia bezpośrednią realizację
struktur sprzętowych i ich równoległe zwielokrotnienie.\cite{peetermans2019portable}

W przeciwieństwie do rozwiązań programowych, implementacja generatora
w strukturze FPGA pozwala na kontrolę nad topologią, sposobem próbkowania
oraz logiką post-processingu. Jest to istotne, ponieważ zarówno jakość
źródła entropii, jak i mechanizm jej ekstrakcji wpływają na końcową
losowość strumienia bitów.\cite{ma2014entropy}

W praktyce platformy FPGA są często wybierane w zastosowaniach
wymagających wysokiej przepustowości i deterministycznej latencji,
a także tam, gdzie istotna jest przenośność rozwiązania pomiędzy
różnymi rodzinami układów. Przykładem są współczesne konstrukcje TRNG
projektowane tak, aby zachowywać właściwości statystyczne niezależnie
od producenta i szczegółów technologii.\cite{matuszewski2024startstop}
