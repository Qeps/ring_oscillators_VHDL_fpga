Do realizacji projektu wybrano środowisko Xilinx Vivado Design Suite,
które stanowi podstawowe narzędzie do projektowania, syntezy oraz
implementacji układów FPGA firmy Xilinx. Vivado umożliwia opis logiki
sprzętowej w językach HDL, przeprowadzenie syntezy logicznej,
rozmieszczenia i trasowania zasobów, a także generację pliku konfiguracyjnego
(bitstream) i programowanie układu docelowego.

Istotną zaletą środowiska Vivado jest pełna kontrola nad strukturą
projektowanego układu oraz dostęp do raportów czasowych i zasobowych,
co ma kluczowe znaczenie w przypadku oscylatorów pierścieniowych.
Niewielkie różnice w opóźnieniach propagacji bramek oraz sposób
rozmieszczenia logiki w strukturze FPGA mają bezpośredni wpływ na
właściwości generowanego sygnału i poziom entropii.

Jako platformę sprzętową wybrano płytkę rozwojową Arty S7--25 z układem
Xilinx Spartan-7. Jest to nowoczesny układ FPGA przeznaczony do aplikacji
o umiarkowanej złożoności, oferujący dobrą relację dostępnych zasobów
logicznych do poboru mocy i kosztu. Rodzina Spartan-7 zapewnia stabilne
parametry czasowe oraz powtarzalne zachowanie struktur logicznych, co
jest istotne w kontekście implementacji generatorów liczb losowych.

Dodatkowym argumentem przemawiającym za wyborem płytki Arty S7--25 jest jej
szerokie wsparcie narzędziowe oraz dostępność dokumentacji i przykładów
projektowych. Płytka posiada standardowy interfejs JTAG do programowania
i debugowania, co umożliwia wygodne testowanie implementacji oraz
obserwację zachowania układu w warunkach rzeczywistych.
