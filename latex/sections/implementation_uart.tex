W niniejszym podrozdziale przedstawiono implementację protokołu UART, przeznaczoną do 
transmisji ciągu losowych bitów generowanych przez moduł \texttt{trng\_ro}. Celem tego etapu 
projektu jest umożliwienie niezależnej, asynchronicznej komunikacji pomiędzy układem FPGA a 
systemem zewnętrznym, w którym dane losowe mogą być dalej przetwarzane lub analizowane.

\begin{lstlisting}[language=VHDL,
caption={Moduł bufora UART (uart\_buff.vhdl)},
label={lst:uart_buff}]
library IEEE;
use IEEE.STD_LOGIC_1164.ALL;

entity uart_buff is
    generic (
        
    );
    port (
        
    );
end uart_buff;
\end{lstlisting}

Listing~\ref{lst:uart_buff} przedstawia architekturę pośrednią. Z uwagi na bitowy charakter 
wyjścia generatora \texttt{trng\_ro} moduł \texttt{uart\_buff} stopniowo agreguje pojedyncze bity
do postaci bajtów, zgodnej z formatem transmisji UART. Rozwiązanie to oddziela 
warstwę generacji entropii od warstwy komunikacyjnej, upraszczając zarówno projekt, jak i 
późniejszą weryfikację poszczególnych bloków funkcjonalnych.
Implementacja została podzielona na dwa współpracujące moduły: bufor danych odpowiedzialny 
za gromadzenie bitów losowych oraz nadajnik UART realizujący właściwy proces transmisji 
szeregowej. Takie rozdzielenie funkcjonalności pozwala na zachowanie czytelnej struktury 
projektu oraz ułatwia ewentualną rozbudowę lub modyfikację parametrów transmisji w dalszych 
etapach pracy.
W kolejnych fragmentach podrozdziału zaprezentowano implementację poszczególnych modułów 
w języku VHDL, a następnie omówiono ich rolę w kontekście całego toru transmisji danych losowych.
