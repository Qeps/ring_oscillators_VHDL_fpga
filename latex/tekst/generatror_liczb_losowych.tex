\chapter{Generator liczb losowych w technice cyfrowej}

Celem niniejszego rozdziału jest omówienie koncepcji cyfrowego generatora
liczb losowych, ze szczególnym uwzględnieniem źródeł entropii oraz
sprzętowej realizacji układu. Przedstawiono rolę oscylatora
pierścieniowego jako elementu wprowadzającego niedeterministyczność oraz
znaczenie platformy FPGA dla jakości i powtarzalności implementacji.

\section{Oscylator pierścieniowy jako źródło entropii}

Oscylator pierścieniowy (ang. \emph{ring oscillator}) jest jednym z
najprostszych układów generujących sygnał okresowy w technologiach
cyfrowych. Składa się z nieparzystej liczby inwerterów połączonych
szeregowo w zamkniętą pętlę, gdzie wyjście ostatniego elementu jest
podłączone do wejścia pierwszego.

Zasada działania jest intuicyjna: sygnał logiczny krążący w pętli jest
kolejno odwracany przez każdy inwerter. Ponieważ liczba inwerterów jest
nieparzysta, układ nie może osiągnąć stabilnego stanu logicznego.
Zamiast tego powstaje ciągła oscylacja pomiędzy stanem wysokim i niskim.
Częstotliwość tych oscylacji zależy głównie od sumy opóźnień propagacji
poszczególnych bramek.

Istotną cechą oscylatora pierścieniowego jest jego podatność na zjawiska
fizyczne zachodzące w strukturze krzemowej. Szumy termiczne, fluktuacje
napięcia zasilania, zmiany temperatury oraz lokalne różnice
technologiczne powodują losowe zmiany opóźnień propagacji. Skutkiem tego
jest nieregularność okresu sygnału, określana jako jitter, który stanowi
praktyczne źródło entropii w wielu konstrukcjach TRNG.\cite{sunar2006provably,saarinen2021entropy}

Losowy charakter jittera sprawia, że oscylator pierścieniowy może pełnić
rolę źródła entropii w cyfrowych generatorach liczb losowych. W praktyce
entropia pozyskiwana jest poprzez próbkowanie sygnału oscylatora lub
porównywanie faz i częstotliwości wielu niezależnych oscylatorów.
Dobór parametrów próbkowania i architektury ma bezpośredni wpływ na
entropię na bit i odporność na zakłócenia.\cite{ma2014entropy}

Dzięki prostocie, niewielkim wymaganiom sprzętowym i dobrej skalowalności
w nowoczesnych procesach technologicznych, oscylatory pierścieniowe są
powszechnie wykorzystywane jako elementy generatorów liczb losowych w
układach FPGA, mikrokontrolerach oraz systemach kryptograficznych.

\section{Implementacja generatora w strukturach FPGA}

Układy FPGA (ang. \emph{Field Programmable Gate Array}) stanowią
elastyczną platformę do implementacji cyfrowych generatorów liczb
losowych, w tym konstrukcji bazujących na jitterze oscylatorów
pierścieniowych. Architektura oparta na programowalnych blokach logicznych
i rekonfigurowalnych połączeniach umożliwia bezpośrednią realizację
struktur sprzętowych i ich równoległe zwielokrotnienie.\cite{peetermans2019portable}

W przeciwieństwie do rozwiązań programowych, implementacja generatora
w strukturze FPGA pozwala na kontrolę nad topologią, sposobem próbkowania
oraz logiką post-processingu. Jest to istotne, ponieważ zarówno jakość
źródła entropii, jak i mechanizm jej ekstrakcji wpływają na końcową
losowość strumienia bitów.\cite{ma2014entropy}

W praktyce platformy FPGA są często wybierane w zastosowaniach
wymagających wysokiej przepustowości i deterministycznej latencji,
a także tam, gdzie istotna jest przenośność rozwiązania pomiędzy
różnymi rodzinami układów. Przykładem są współczesne konstrukcje TRNG
projektowane tak, aby zachowywać właściwości statystyczne niezależnie
od producenta i szczegółów technologii.\cite{matuszewski2024startstop}
